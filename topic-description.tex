\documentclass[a4paper,12pt]{article}

\usepackage{titlesec}
\titleformat*{\section}{\large\bfseries}
\titleformat*{\subsection}{\normalsize\bfseries}

\pagestyle{empty}

%\usepackage[ngerman]{babel}

\usepackage[utf8]{inputenc}
\usepackage[top=0.8in, bottom=0.8in, left=0.8in, right=0.8in]{geometry}

\usepackage{graphicx}
\usepackage{url}
\usepackage{paralist}
\usepackage{lmodern}
\usepackage[authoryear]{natbib}
\usepackage{blindtext}
\renewcommand\textbullet{\ensuremath{\bullet}}

\renewcommand{\familydefault}{\sfdefault}

\date{}

\title{
%\vspace*{-2cm} \includegraphics[width=16cm]{header.png}
\includegraphics[width=6cm]{figures/stuttgart-vector.pdf}\hfill{\includegraphics[width=3cm]{figures/sqa_logo.png}}
\quad \\ [0.5cm]
{\large \textit{Bachelor's Thesis (Bachelor Softwaretechnik):}} \\ [1mm]
{\Large Simulating Scenario-based Chaos Experiments for Microservice Architectures}
}

\begin{document}
	

\maketitle

\thispagestyle{empty}

\vspace{-2.5cm}


\subsection*{Background and Motivation}
In a preceding case study we showed that scenario-based chaos tests are a valid way for testing the resilience of a microservice architecture \cite{CaseStudyDatev}. Besides running such scenarios in production, they could also be simulated to increase time efficiency and safety of such tests. Therefore, a preliminary requirements specification for resilience scenario simulators was created. After a superficial evaluation of five existing microservice simulators it came clear that none of these currently support the requirements to a completely level. 
However, some simulators provide sophisticated tools for specific types of scenarios. For example, SimuLizar \cite{SimuLizar1} has very advanced work load description tools, whilst MiSim \cite{MiSim} can simulate chaos like fault loads. 

\subsection*{Goals}
In the scope of this thesis a requirement analysis will take place to establish the exact requirements for a resilience scenario simulator. This will include doing interviews and questionings with the known stakeholders.
Further, based on these requirements existing microservice simulators will be evaluated with the help of developer interviews, code analysis and a structured research. The goal is to find out which scenario type each simulator can simulate the best.
The most versatile or promising (with regard to the requirements) simulators will then be extended to support a common and formal scenario description, which is being developed in another project.
Lastly, the extended simulators will be used to simulate actual scenarios, created and executed in the preceding case study \cite{CaseStudyDatev}. Results of the simulation will then be compared to the real results and show how accurate the simulators are.

\subsection*{Possible Collaborations}
As aforementioned this thesis will reuse the data gathered by a preceding case study. Further, we are currently looking into another possible collaboration, which would supply us with further scenarios and real world data. 
Lastly, there are three ongoing thesis by other students that also  revolve around the

\begin{scriptsize}
\bibliographystyle{abbrv}
\bibliography{bibliography.bib}
\end{scriptsize}

\subsection*{Contact}
Dr.-Ing. André van Hoorn, van.hoorn@informatik.uni-stuttgart.de \\
University of Stuttgart, Inst.\ for Software Engineering, Software Quality and Architecture Group \\

\end{document}

